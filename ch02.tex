\chapter{Командний рядок}

Я уявляю свого читача, який зараз відкрив термінал, розгорнув його на весь екран та збирається з думками.
Майже увесь екран чорний, лише у куточку виведено запрошення та блимає курсор.
Цей читач вдивляється в цю пустелю, у цей войд, там задається питаннями:
«Нащо я сюди зайшов?
Як тут не пропасти?
Навіщо це мені треба???»

Можливо навпаки, хтось розгорнув на увесь монітор білий екран терміналу у Mac~OS, та читає в нетерплячці:
«Дайте мені завдання!
Зараз швидко розберуся, та піду рухатися далі!»

Здорова оцінка ще ніколи в IT не заважала, побоювання першого читача обґрунтовані.
Тому ми почнемо цю главу з мотиваційного параграфу, а саме з описом переваг, який надає командний рядок.

\vfill
\pagebreak

\section{Мотивація}

Командний рядок виглядає ворожим оточенням для непідготовленого користувача.
Усе незвично, усе лякає, з чого починати?
Звісно що з мотивації, бо це одна з головних речей у навчанні.
Коли ви розумієте, навіщо ви це робите, які цілі досягаєте, коли бачите світло наприкінці тунелю, то й рухатися уперед значно легше.
А коли ви кружляєте у темряві без гадки про те, чого сподіваєтеся знайти, то скільки ви можете протриматися?
Особисто я недовго.

Але ворога треба знати у лице, тому для початку спробуємо найти переваги, які є у графічному інтерфейсі користувача (Graphical User Interface, або GUI).\index{GUI}
Головне, що дає GUI, це розуміння того, які дії ми можемо виконати.
Перед нами є меню, де ви приблизно знаєте, де шукати потрібні операції.
Наприклад, вихід з програми зазвичай у меню «File», а там за обставинами це може бути «Exit» або «Quit».
Якщо ви хочете зробити дію з якимось елементом, що можна спробувати викликати контекстне меню через натискання правої кнопки миші.
Так, інколи потрібний функціонал дуже сильно прихований, інколи його переносять з одного місця на інше у новій версії продукту, але у цілому це працює.

У командному рядку нам треба набирати команди.
Навіть для типових команд можуть існувати багатого різних можливих варіантів.
Наприклад, інколи, щоб вийти з програми, нам треба ввести \cmdd{exit}, а інколи це \cmdd{quit}.
Для видалення можуть використовуватися такі команди, як \cmdd{remove}, \cmdd{delete}, \cmdd{erase}, або, навіть, скорочення \cmdd{rm} та \cmdd{del}.
Той факт, що треба пам'ятати, яку саме команду треба вводити в тій чи іншій програмі, трохи напружує---треба читати мануали, що більшість користувачів, м'яко кажучи, не полюбляє.
Можливо, коли вам треба один раз на рік, два рази на Пасху щось виправити у зображенні, то дійсно краще запустити графічний редактор, у якому ви зможете знайти необхідні дії без того, щоб ритися у документації.
Але, якщо ви збираєтеся йти у програмування, то хочеш не хочеш---мусиш мати звичку читати документацію.
Тому, як на мене, до цього краще потроху звикати, шкідливою така звичка 100\% не буде.

А тепер поговоримо про переваги командного рядку, та чому багато розробників та системних адміністраторів його полюбляють.
Розглянемо на конкретному прикладі зміни атрибутів файлу.
У GUI я маю виконати наступну послідовність дій: відкрити програму перегляду файлової системи, знайти файл \file{install.sh}, вибрати його, викликати контекстне меню, там «Властивості», далі перейти на вкладику «Доступ», та виставити галочку «Дозволити виконання як програми».
У командному рядку треба набрати та виконати \cmdd{chmod +x install.sh}.

Що відрізняється?
По-перше, це кількість тексту, який описує необхідні дії.
Щоб описати послідовність дій у GUI ми написали невеличке оповідання.
Але проблема не тільки у кількості слів, але й у тому, наскільки повний цей опис.
У разі командного рядку ми маємо команду, яка виглядає крізь однаково, яку досить легко набрати, або, ще краще, скопіювати, та виконати.
Якщо брати випадок GUI, то різні версії програми можуть мати різні назви (як ми пам'ятаємо, це не суттєво для користувача, бо він зазвичай бачить усі альтернативи та може вибрати відповідну).
Різні версії програмного забезпечення можуть перевпорядкувати розташування елементів, що призводить ти ситуації, коли ти пам'ятаєш, що ця функція має бути, але її знову розшукати.
Також програми можуть мати різні налаштування, що призводить до того, що одна й та сама послідовність дій може призводити до різних результатів.
Ну і... спробуйте нагуглити \google{chmod +x}, та придумайте відповідний запит для GUI застосунку.

\begin{exercise}
Перевірте видачу Google за запитом \google{chmod +x}, та спробуйте продивитися декілька результатів, які вас зацікавлять.
\end{exercise}

По-друге, це звіти про помилки.
Наприклад, інколи спроба виконання команди \cmdd{chmod +x install.sh} може призвести до помилки\footnote{
Зверніть увагу, що ми отримали повідомлення на англійській мові.
\todo{reference to intro}
}:

\begin{Verbatim}[fontsize=\footnotesize]
$ chmod +x install.sh
chmod: changing permissions of 'install.sh': Operation not permitted
\end{Verbatim}

Але, ми принаймні спробували виконати команду, та отримали повідомлення, чому саме ця дія не може бути виконана.
На українську мову це повідомлення можна перекласти як «заміна прав для файлу \file{install.sh}: операція не дозволена».
Ми можемо загуглити\footnote{Наприклад, \google{chmod Operation not permitted}} чому саме ця операція не може бути виконана.
У GUI часто ми можемо побачити, що певні дії нам недоступні, потрібний нам прапорець сірий.
Але ніякої підказки, чому саме вони недоступні, зазвичай немає.

А чи бувало у вас таке, що ви щось робили давно, але лише один раз,
хтось вам підказав послідовність дій, і ви вже забули як?
Чи траплялося, що ви знайшли рішення складної проблеми,
але через місяць вже не пам'ятаєте, що саме робили?
У командному рядку є чудова можливість---історія команд.
Усі команди, які ви виконували, зберігаються, і до них можна повернутися.
Можна переглянути історію за допомогою команди \cmdd{history}.
Це особистий записничок усіх ваших дій, який ведеться автоматично.
Спробуйте уявити таку функціональність у GUI.
Можна зберігати відео з екрана, але це гігабайти інформації,
плюс спробуйте потім там щось знайти!

Завжди корисно, а особливо коли ви перші дні на новій роботі
й треба багато чому навчитися, вести свій особистий записничок.
Неважливо, чи ви працюєте у GUI, чи у командному рядку.
Але у разі GUI нам треба писати купу тексту на кшталт
«відкрити меню Правка, потім Налаштування, вкладка Додатково, прапорець біля опції XYZ».
У командному рядку достатньо просто скопіювати\footnote{
Тут може чекати невеличка неприємність, бо сполучення \keys{\ctrl}~+~\keys{C},
до якого звикла переважна більшість користувачів, скоріше за все не буде працювати.
Причина полягає у тому, що термінали з'явилися задовго до виникнення буферу обміну,
тому ця комбінація клавіш використовувалася для іншої задачі.
Але має працювати клік колесиком (або клік правою клавішею) миші,
або, наприклад, \keys{\shift}~+~\keys{Ins}.
} корисну команду з коротким коментарем.
А коли треба звернутися до колеги за допомогою,
зазвичай команда набагато інформативніша за скріншот.

У часи, коли LLM стали невід'ємною частиною нашого життя,
використання командного рядку отримало ще одну перевагу.
Коли ви запитуєте ChatGPT або Claude «як зробити X у командному рядку»,
відповідь зазвичай містить декілька готових команд, яку можна скопіювати та виконати.
Текст це рідна мова для LLM.
Не зважаючи на те, що LLM може також описати послідовність дій у GUI,
працювати з текстом команд простіше.
Ми можемо легко задати уточнювальні питання,
запропонувати змінити частину команди,
попросити пояснити окремі параметри.
AI може швидко модифікувати команду під ваші потреби---
замінити назву файлу, додати додаткові опції, поєднати з іншими командами.

\begin{exercise}
Запитайте у ChatGPT або Claude, як знайти усі файли з розширенням \file{.txt}
у поточній папці за допомогою командного рядку.
Далі попросіть виключити з пошуку директорію \file{.git}.
\end{exercise}

Добре, ліричний відступ закінчено.
Сподіваюся, цього достатньо, щоб у вас виникло бажання
спробувати командний рядок та оцінити його переваги на власному досвіді.
Рухаємося далі.

\section{Початок роботи}

Гадаю, що після попереднього параграфу ви достатньо мотивовані, щоб рухатися далі.
