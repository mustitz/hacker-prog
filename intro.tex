\chapter*{Вступ}

\section*{Структура книги}

Перед тим, як розпочати важкий шлях навчання, я хочу сказати декілька слів щодо побудови книги, та дати декілька порад щодо самого процесу навчання.
Але треба розуміти, що цей вибір ґрунтується на моїй точці зору, до якої, як, в принципі, і до усього, треба ставитися критично.
Якщо ви відчуваєте у собі сили йти наперекір---чому б й ні?
Усі люди індивідуальні, тому будь які узагальнення мають свої виключення.
Ви можете бути як раз цим виключенням.
Більше того, таким виключенням можу бути я.

І першою порадою буде дочитати цей параграф, щоб хоча б мати уяву, що очікувати та які є варіанти.

\subsection*{Мова повідомлень}

Усі повідомлення, які виводить система, у книзі будуть наведені англійською мовою.
Так, є можливість налаштувати систему таким чином, щоб повідомлення виводилися українською мовою.
Багато дистрибутивів послужливо встановлюють українську мову крізь де тільки можливо відразу після того, як дізнаються про ваше географічне місце знаходження.
Але моя порада все ж таки встановити англійську локалізацію хоча б для командного ряду.
\todo{reference to appendix with explanation how to do it}

Хочете ви того чи ні, але міжнародна мова в IT це англійська.
Тому, якщо ви її не знаєте, то варіантів небагато: хочеш не хочеш---мусиш вийти хоча б на початковий рівень.
Як не крути, але англомовне ком'юніті у світі набагато більше, ніж україномовне.
Без англійської кількість доступної інформації суттєво скорочується, а у деяких моментах автоматичний переклад\footnote{
В інтернеті усе більше й більше з'являється автоматично згенерованих сайтів з машинним перекладами документації, які ставлять за мету зібрати якмога більше трафіку.
} може вводити в оману.

Але, з іншого боку, кидати усе й починати вчити англійську також не краща ідея: чим далі ви будете тягнути з програмуванням, тим страшніше буде починати.
Саме й тому я пишу цю книгу української мовою, щоб була можливість принаймні спробувати та зрозуміти, чи є вас хист до цієї діяльності чи немає.
Але й відкладати вивчення англійської в довгий ящик теж не кращий варіант: чим міцніше ви будете прив'язані до української термінології, там важче буде з не потім порвати та переналаштуватися.

Моя порада полягає у тому, що якщо ви новачок, то починайте вивчати програмування та удосконалювати англійську одночасно.
Технічна англійська мова не складна.
Це не спілкуванні у реальному світі, де часто треба брати до увагу ментальність та натяки.
Коли ти спілкуєшся на побаченні на рідній мові, то ти зазвичай орієнтуєшся у просторі можливостей, та відчуваєш перспективи поглиблення знайомства та розвиток подій.
Коли це іноземна мова, то усе стає складніше.
Ця посмішка, ці слова означають симпатію до вас або виключно ввічливість?
Цих дилем зазвичай немає у технічній мові, де зазвичай треба обрати одну з двох розумних альтернатив.
Так, інколи, навіть люди які спілкуються рідною мовою, не розуміють одне одного та роблять неправильний вибір.
Це життя, від цього не втекти.
Тому не впадайте у відчай, коли ви зрозумієте щось неправильно, це цінний досвід який допоможе рухатися далі.

І ось, перший крок на цьому шляху паралельного удосконалення англійської та вивчення програмування як раз і є встановлення англійської локалізації.
\todo{reference to appendix with explanation how to do it}
Які переваги від цього?
По-перше, результати пошуку у Google за текстом повідомлення на англійській мові будуть набагато більш релевантними.
По-друге, як не крути, це буде змушувати вас потроху тренувати англійську.
По-третє, усі повідомлення в цій книзі узяті з англійської локалізації, така ось спроба автора примусити читача дотримуватися його порад.

Чи треба відвідувати курси англійської мови?
Тут я не можу дати ніякої поради, як на мене це дуже особисто.
Можу сказати лише, що мені у житті дуже допомагав метод Фейнмана, описаний ним у книзі «Та ви жартуєте, містере Фейнман!»
Цей метод полягає тому, що коли ми отримаємо незрозуміле повідомлення, то ми зупиняємося та не рухаємося далі, доки його не зрозуміємо.
Особисто я на початку перекладаю усі незнайомі слова, та виписую їх у записничок, щоб краще запам'ятати на майбутнє.
Потім пробую зрозуміти побудову речення, що й як з чим зв'язано.
Тут я часто використовую переклад Google, але інколи скорочую речення заміняючи зрозумілі фрагменти, або місця, які вводять перекладач в оману, на синоніми.
І так експериментую, доки не з'явиться почуття, що усе зрозуміло й можна рухатися далі.
Я гадаю, що читач також сам зможе знайти такі, як кажуть програмісти, хаки, які допоможуть йому невпинно покращувати англійську мову у процесі вивчення програмування.
